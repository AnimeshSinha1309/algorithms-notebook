\chapter{Probability}



\section{Methods of Solving Problems}

All Probability or Expectation Value problems are solvable using these techniques, so trying to classify all problems in one of the following is reasonable.

\subsubsection{Probability times Value (The Definition)}

Expectation by Definition is Probability times the Value of the event. If the state space is ennumerable, then this is the technique to go with. Sometimes, \textbf{Dynamic Programming} can be used with this technique.

\subsubsection{Technique of Contributions (Exchange Argument)}

For each element, find what is the Probability that it contributes to the total answer, and how much does it contribute. Try to solve for every pair of elements, then take it to all the elements.

\subsubsection{Indicator Variables}

Interconverting sums using indicator variables is often a hard and algebraically tedious topic, hence we mark it as a \#TODO.


\section{Problems and Solutions}


\subsection{Codeforces 100371-H: Granite of Science}

\begin{problem}[Granite of Science]{}
    There are $n$ subjects, each having $n_i$ lessons. Let the lessons be indexed by $j$. We are looking at the first $m$ days of the semester. Any subject can start on any of the m days (even if all it's lesson's don't fit till the end), and each lesson will follow consecutively and contigously. Given that $h_{i,j}$ is the fatigue due to i-th subject's j-th lesson, and effective fatigue in one day is the sum of squares of fatigues due to all lessons that day. Total Fatigue is sum of effective fatigue in each of m days. Compute expected total fatigue.
    
    \colorbox{green}{Tags}: 2D Prefix Sum , Linearity of Expectation, Independent Events.
\end{problem}

Following is one solution to this problem (simplest and cleanest). Another one using Indicator Variables also exists, but seems to overcomplicate a relatively simple question.


Notation: $n$ is number of subjects. $f_{\text{day}}$ is the sum of difficulties of all subjects on that day. $f_{\text{day}} = \sum_{i=1}^{i}d_i,j$, where $d_i$ is difficulty of i-th subject for that particular day and j-th lesson. 

\begin{eqnarray*}
    \text{ans} &=& \sum_{\text{day}=1}^{n} f_{\text{day}}^2 \\
    \implies E[\text{ans}] &=&  E\left[\sum_{\text{day}=1}^{2m}f_{\text{day}}^2 \right] \\
    &=& \sum_{\text{day}=1}^{2m}E\left[f_{\text{day}}^2\right]  \\
\end{eqnarray*}

Now, $f_{\text{day}}^2 = (c_1+c_2+\ldots+c_n)^2=(\sum_{i}c_i^2 + 2\sum_{i,j; i\ne j}c_ic_j)$

\begin{eqnarray*}
    E\left[f_{\text{day}}^2\right]&=&\sum_{i}E[c_i^2] + 2\sum_{i,j; i\ne j}E[c_ic_j] \\
    &=& \sum_{i}E[c_i^2] + 2\sum_{i,j; i\ne j}E[c_i]E[c_j] 
\end{eqnarray*}

Because $i$-th and $j$-th subject's contribution on the same day are independent of each other, their expectations can be split. The final expression can be simplified using the following:

\begin{eqnarray*}
    2\sum_{i,j; i\ne j}E[c_i]E[c_j] = \sum_{i,j}E[c_i]E[c_j] - \sum_{i}E[c_i]^2
\end{eqnarray*}

Thus, now we can use prefix sums and solve this problem in $\mathcal{O}(N^2)$.
